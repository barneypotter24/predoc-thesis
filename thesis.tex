\documentclass[showinstructions,faculty=med,department=rega,phddegree=predoc,doctoralschool=bms]{adsphd}
%\documentclass[showinstructions,faculty=firw,department=cws,phddegree=cws,doctoralschool=ads]{adsphd}
%\documentclass[showinstructions,showlabels,coverfontpercent=100,faculty=firw,department=cws,phddegree=cws]{adsphd}
%\documentclass[croppedpdf,showinstructions,faculty=firw,department=cws,phddegree=cws]{adsphd}
%\documentclass[online,faculty=firw,department=cws,phddegree=cws]{adsphd}
%\documentclass[print,bleed,cropmarks,faculty=firw,department=cws,phddegree=cws]{adsphd} % include bleed for the print service
%\documentclass[print,faculty=firw,department=cws,phddegree=cws]{adsphd}
%\documentclass[final,faculty=firw,department=cws,phddegree=cws]{adsphd}
%\documentclass[showinstructions,faculty=firw,department=cws,phddegree=cws,epub]{adsphd}

% !!!!!!!!!!!!!!!!!!!!!!!!!!!!!!!!!!!!!!!!!!!!!!!!!!!!!!!!!!!!!!!!!!!
% !!                                                               !!
% !!  WARNING: do not remove the following lines between           !!
% !!  "%%% COVER: Settings %%%" and "%%% COVER: End settings %%%"  !!
% !!                                                               !!
% !!!!!!!!!!!!!!!!!!!!!!!!!!!!!!!!!!!!!!!!!!!!!!!!!!!!!!!!!!!!!!!!!!!

%%% COVER: Settings %%%
\title{Bayesian inference methodologies to reconstruct the evolution and spread of pathogens}
%\subtitle{Multiscale computing for Dummies}

\author{Barney}{Potter}

\supervisor{Prof.~Dr.~G.~Baele}{}
\supervisor{Dr.~M.S.~Gill}{}
\president{Prof.~Dr.~G.~Opdenakker}
\jury{Prof.~Dr.~P.~Lemey\\
      Prof.~Dr.~J.~Matthijnssens\\
}
\externaljurymember{Prof.~Dr.~L.~Dupont}{Secretary}

% Your research group within the department
% e.g. iMinds-DistriNet, Scientific Computing Group, ...
\researchgroup{Computational Evolutionary Virology}
\website{} % Leave empty to hide
\email{} % Leave empty to hide

\address{Herestraat 49}
\addresscity{3000 Leuven} % This is the default value. Note
                              % that 'B-3001 Heverlee' is _incorrect_!!
                              % /[https://www.kuleuven.be/communicatie/marketing/intranet/huisstijl/taalgebruik.html]

\date{August 2020}
\copyyear{2020}
%\udc{XXX.XX}            % UDC, deposit number and ISBN are no longer necessary.
%\depot{XXXX/XXXX/XX}    % Leave out the initial D/ (it is added
                         % automatically)
%\isbn{XXX-XX-XXXX-XXX-X}


% Set spine width:
\setlength{\adsphdspinewidth}{9mm}

%% Set bleeds
%\setlength{\defaultlbleed}{7mm}
%\setlength{\defaultrbleed}{7mm}

% Set custom cover page
% \setcustomcoverpage{mycoverpage.tex} % mycoverpage.tex is the default

%%% COVER: End settings %%%

% for the nomenclature (comment out if you do not use the nomencl package
\usepackage{nomencl}   % For nomenclature
\renewcommand{\nomname}{List of Symbols}
\newcommand{\myprintnomenclature}{%
  \cleardoublepage%
  \printnomenclature%
  \chaptermark{\nomname}
  \addcontentsline{toc}{chapter}{\nomname} %% comment to exclude from TOC
}
\makenomenclature%

% for the glossary (comment out if you do not use the glossaries package)
\usepackage{glossaries} % For list of abbreviations
\newcommand{\glossname}{List of Abbreviations}
\newcommand{\myprintglossary}{%
  \renewcommand{\glossaryname}{\glossname}
  \cleardoublepage%
  \printglossary[title=\glossname]
  \chaptermark{\glossname}
  \addcontentsline{toc}{chapter}{\glossname} %% comment to exclude from TOC
}
\makeglossary%

% For highlighting comments in red so that they are easily seen/removed later:
\newcommand{\barneycomment}[1]{\textcolor{blue}{#1}}
\newcommand{\guycomment}[1]{\textcolor{red}{#1}}

% BibLaTeX
%\usepackage[utf8]{inputenc}
%\usepackage{csquotes}
%\usepackage[
  %hyperref=auto,
  %mincrossrefs=999,
  %backend=biber,
  %style=authoryear-icomp
%]{biblatex}
%\addbibresource{allpapers.bib}

% Fonts
\usepackage{textcomp} % nice greek alphabet
\usepackage{pifont}   % Dingbats
\usepackage{booktabs}
\usepackage{amssymb,amsthm}
\usepackage{amsmath}


%%%%%%%%%%%%%%%%%%%%%%%%%%%%%%%%%%%%%%%%%%%%%%%%%%%%%%%%%%%%%%%%%%%%%%

\begin{document}

%%%%%%%%%%%%%%%%%%%%%%%%%%%%%%%%%%%%%%%%%%%%%%%%%%%%%%%%%%%%%%%%%%%%%%

\makefrontcoverXII

\maketitle

\frontmatter % to get \pagenumbering{roman}

% \includepreface{preface}
\includeabstract{summary}
% \includeabstractnl{abstractnl}

% To create a list of abbreviations, there are 2 options
% 1. manual creation of list of abbreviations and inclusion as a chapter
%    \includeabbreviations{abbreviations}
% 2. automatic generation via the glossary package
%    \glossary{name=MD,description=molecular dynamics}
\myprintglossary

% To create a list of symbols, there are 2 options
% 1. include a manually created nomenclature as a chapter
%    \includenomenclature{nomenclaturechapter}
% 2. automatic generation via the nomencl package
%    \nomenclature[cB]{$c_B(\vec{x})$}{Characteristic function of $B$}
% \myprintnomenclature

\tableofcontents
\listoffigures
% \listoftables

%%%%%%%%%%%%%%%%%%%%%%%%%%%%%%%%%%%%%%%%%%%%%%%%%%%%%%%%%%%%%%%%%%%%%%

\mainmatter % to get \pagenumbering{arabic}

% Show instructions on a separate page
% \instructionschapters\cleardoublepage

% Put all glossary definitions in one location
\newglossaryentry{eids}{name={EIDs},description={Emerging infectious diseases. A number of infectious diseases, many of whom are caused by viruses, that have grown significantly in human impact in recent years (e.g. Zika, Ebola, Lassa)}}
\newglossaryentry{hbv}{name={HBV},description={Hepatitis B virus. The causative agent of Hepatitis B, a liver disease infecting millions worldwide. One of two viruses being analyzed as part of this thesis, along with LASV}}
\newglossaryentry{lasv}{name={LASV},description={Lassa virus. The causative agent of Lassa fever, a hemorrhagic fever that affects hundreds of thousands of people annually, primarily in sub-Saharan Africa. One of two viruses being analyzed as part of this thesis, along with HBV}}
\newglossaryentry{ess}{name={ESS},description={Effective sample size. Metric representing the number of functionally independent samples are present in a posterior sample set}}
\newglossaryentry{beast}{name={BEAST},description={Bayesian Evolutionary Analysis Sampling Trees. Software package implementing MCMC and related methods used for Bayesian phylogenetic inference in this thesis}}
\newglossaryentry{mrca}{name={MRCA},description={Most recent common ancestor. The most recent organism from which a set of contemporary organisms are descended}}
\newglossaryentry{mep}{name={MEP},description={Measurably evolving population. A set of organisms whose evolution can be directly observed over a period of time, and for which this evolutionary rate can be quantified as a function of time}}
\newglossaryentry{mcmc}{name={MCMC},description={Markov chain Monte Carlo. A class of algorithms in which a target posterior distribution is parameterized (as a Markov chain), explored, and recorded}}
\newglossaryentry{mcc}{name={MCC},description={Maximum clade credibility. Given a set of posterior trees, the individual tree whose internal nodes (``clades'') appear in the highest proportion of the entire distribution of trees}}
\newglossaryentry{ctmc}{name={CTMC},description={Continuous-time Markov chain. Markov process in which jumps between states are parameterized in terms of their ``holding time'' in states as a continuous variable, rather than as discrete-time jump probabilities}}
\newglossaryentry{bssvs}{name={BSSVS},description={Bayesian stochastic search variable selection. A method by which parameters' inclusion in a model are ``toggled'' randomly to determine whether those parameters should be considered part of the model}}
\newglossaryentry{bf}{name={BF},description={Bayes factor. The likelihood ratio of the marginal likelihood of two different hypotheses}}
\newglossaryentry{dag}{name={DAG},description={Directed acyclic graph. A graphical representation of the ``flow'' of a Snakemake script in which each steps leads out of the previous and into the next, and for which there are no ``loops'' backwards in the process}}
\newglossaryentry{hmc}{name={HMC},description={Hamiltonian Monte Carlo. A method of MCMC whereby parameters are modified as a group by treating their movement through likelihood space as a single particle with inertia}}
\newglossaryentry{tmrca}{name={tMRCA},description={Time of the most recent common ancestor. Point in time inferred by phylogenetic inference at which the MRCA was hypothesized to have existed}}

\includechapter{introduction}
% \includechapter{manual} % Remove this chapter

% Insert here your own chapters
% Chapters are expected to be in a tex-file with the given name dot
% tex and in a directory with the given name in the chapters
% directory.
\includechapter{objectives}
\includechapter{methodology}
\includechapter{results}
\includechapter{discussion}
% \includechapter{conclusion}


%%%%%%%%%%%%%%%%%%%%%%%%%%%%%%%%%%%%%%%%%%%%%%%%%%%%%%%%%%%%%%%%%%%%%%

\appendix

\includeappendix{papersappendix}

%%%%%%%%%%%%%%%%%%%%%%%%%%%%%%%%%%%%%%%%%%%%%%%%%%%%%%%%%%%%%%%%%%%%%%
\backmatter

\includebibliography
% BibTex
\bibliographystyle{acm}
\bibliography{allpapers}
% BibLatex (comment lines above and comment out biblatex lines in preamble)
%\printbibliography[title=\bibname]
% \instructionsbibliography

% \includecv{curriculum}

% \includepublications{publications}

\makebackcoverXII

\end{document}
